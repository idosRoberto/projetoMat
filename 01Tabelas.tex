\documentclass[a4paper,12pt]{article}
\usepackage[top=2cm, bottom=2cm, left=2.5cm, right=2.5cm]{geometry}

\usepackage[T1]{fontenc}
\usepackage[utf8]{inputenc}
\usepackage[brazilian]{babel}
\usepackage{setspace}
\usepackage{indentfirst}
\usepackage{amsmath,array,amssymb} %Pacotes para poder usar o pmatrix
\usepackage{graphicx} % Pacote para inserir imagens

\DeclareMathOperator{\sen}{sen}

%===CBEÇALHO===========================================
\title{\textbf{TABELAS}}
\author{Autor: Roberto Farias}
\date{Data: 21/Maio/2023}
%======================================================

\begin{document}
\maketitle % Faz com que o cabeçalho apareça no PDF

\textbf{As tabelas abaixo foram criadas usando o ambiente tabular}
\vspace{1cm}

\noindent
\textbf{Tabela 01: Tabela simples}\\
\begin{tabular}{|l|c|r|} \hline
	Célula1 & Célula2 & Célula3\\ \hline
	Célula4 & Célula5 & Célula6\\ \hline
\end{tabular}

\vspace{1cm}

\noindent
\textbf{Tabela 02: mesclando colunas}\\
\begin{tabular}{|c|c|c|c|} \hline
	\multicolumn{4}{|c|}{Meses Do Ano}\\ \hline
	Janeiro & Fevereiro & Março & Abril \\ \hline
	Maio & Junho & Julho & Agosto \\ \hline
	Setembro & Outubro & Novembro & Dezembro \\ \hline
\end{tabular}

\vspace{1cm}

\noindent
\textbf{Tabela 03: Mesclando Segunda e Terceira Colunas}\\
\begin{tabular}{|c|cc|}\hline
Números: & \multicolumn{2}{c|}{Meses Do Ano Abreviações}\\ \hline
1 & Janeiro & Jan \\ \cline{2-3}
2 & Fevereiro & Fev \\ \cline{2-3}
3 & Março & Mar \\ \cline{2-3}
4 & Abril & Abr \\ \cline{2-3}
5 & Maio & Mai \\ \cline{2-3}
6 & Junho & Jun \\ \hline
\end{tabular}

\vspace{1cm}
\textbf{As tabelas abaixo foram criadas usando o ambiente table}
\begin{table}[h]
\centering
\caption{Os Maiores Países Do Mundo Em Extensão}
\vspace{0.5cm}
	\begin{tabular}{c|cc}
		Posição & Países & Extenção Territorial $Km^2$\\ \hline
		1 & Rússia & 17.089.246\\
		2 & Canadá & 9.984.670\\
		3 & China & 9.596.961\\
		4 & Estados Unidos & 9.371.174\\
		5 & Brasil & 8.515.767
	\end{tabular}
\end{table}

\end{document}