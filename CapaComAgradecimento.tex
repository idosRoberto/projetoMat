\documentclass[a4paper,12pt, oneside]{book}
\usepackage[top=2cm, bottom=2cm, left=2.5cm, right=2.5cm]{geometry}

\usepackage[T1]{fontenc}
\usepackage[utf8]{inputenc}
\usepackage[brazilian]{babel}
\usepackage{setspace}
\usepackage{indentfirst}
\usepackage{amsmath,array,amssymb} %Pacotes para poder usar o pmatrix
\usepackage{graphicx} % Pacote para inserir imagens

\DeclareMathOperator{\sen}{sen}

\begin{document}

\begin{titlepage}
	\addtolength{\topmargin}{1.5cm} % Espaço do topo até o texto
	\setlength{\baselineskip}{1.4\baselineskip} % Espaço
	
	\begin{center}
		{\large{NOME DA UNIVERSIDADE}}
		
		{\large{INSTITUIÇÃO ACADÊMICA OU ESCOLA OU FACULDADE}}
	\end{center}

	\vspace{2cm}
	
	\begin{center}
		{\Large{\textbf{Título Do Seu Trabalho}}}
	\end{center}		
	
	\vspace{1.5cm}
	
	\begin{center}
		{\Large{Autor Do Trabalho}}
	\end{center}
	
	\vspace{2cm}
	
	\begin{flushright}
		\begin{minipage}[t]{12cm}
			\hrulefill
	
				Trabalho Final de graduação do curso de Análise e desenvolvimento de 	Sistemas da Faculdade de Tecnologia do Estado de São Paulo apresentado como requisito para a obtenção do grau de tecnólogo em Análise e desenvolvimento de sistemas.

			\hrulefill
			
			\vspace{0.2cm}
			
			{\bf Orientador: Prof. Dr. Fulano}
		\end{minipage}
	\end{flushright}
%------------------------------------------------------	
	\setlength{\baselineskip}{0.7\baselineskip}
	\vfill %Coloca informações na base da folha

	\begin{center}
		Sergipe
		
		Maio de 2023
	\end{center}
	
\end{titlepage}

%------------------------------------------------------
%Resumos e Agradecimentos

\chapter*{Resumo}
\noindent
Aqui vai o resumo do seu trabalho\\
{\textbf{Palavras-chave:}Matemática,geometria,trigonometria}

\chapter*{Agradecimentos}
Aqui vai os agradecimentos

\end{document}