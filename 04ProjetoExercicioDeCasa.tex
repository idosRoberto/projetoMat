\documentclass[a4paper,12pt]{article}
\usepackage[top=2cm, bottom=2cm, left=2.5cm, right=2.5cm]{geometry}

\usepackage[T1]{fontenc}
\usepackage[utf8]{inputenc}
\usepackage[brazilian]{babel}
\usepackage{setspace}
\usepackage{indentfirst}
\usepackage{amsmath,array,amssymb} %Pacotes para poder usar o pmatrix

\DeclareMathOperator{\sen}{sen}

%===CBEÇALHO===========================================
\title{\textbf{Exercício De Fixação - Curso De Latex}}
\author{Autor: Roberto Farias}
\date{Data: 21/Maio/2023}
%======================================================

\begin{document}
\maketitle % Faz com que o cabeçalho apareça no PDF

Abaixo, seguem os exercícios resolvidos:

\vspace{1cm}

\textbf{Exercício 01:}
\begin{equation}
	\sqrt[3]{\left(\frac{2^3+2^5}{10}\right)}
\end{equation}

\textbf{Exercício 02:}
\begin{equation}
	\overline{(x \cdot y)^4}=\overline{x^4} \cdot \overline{y^4}
\end{equation}

\textbf{Exercício 03:}
\begin{equation}
	\frac{a}{sen \widehat{A}}=\frac{B}{sen \widehat{B}}=\frac{c}{sen \widehat{C}}=2r
\end{equation}

\textbf{Exercício 04:}
\begin{equation}
	\vert \vec{u} \times \vec{v} \vert = \vert \vec{u} \vert \cdot \vert \vec{v} \vert \cdot sen (\theta)
\end{equation}

\textbf{Exercício 05:}
\begin{equation}
	\frac{1}{\left(\frac{2}{3} \: cm/s \right)^2} \frac{\partial^2 \psi}{\partial t^2}-\frac{\partial^2 \psi}{\partial x^2}=0
\end{equation}

\end{document}