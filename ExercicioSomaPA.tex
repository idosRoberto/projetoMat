\documentclass[a4paper,12pt, twocolumn]{article}
\usepackage[top=2cm, bottom=2cm, left=2.5cm, right=2.5cm]{geometry}

\usepackage[T1]{fontenc}
\usepackage[utf8]{inputenc}
\usepackage[brazilian]{babel}
\usepackage{setspace}
\usepackage[shortlabels]{enumitem}
\usepackage{mathtools}
\usepackage{amsmath,amsfonts,amssymb}
\usepackage{nopageno}

\begin{document}
\title{Exercício: Soma Dos Termos De Uma PA}
\author{Professor: Roberto Farias; Disciplina: Matemática}
\date{}
\maketitle
\pagestyle{empty}

\begin{enumerate}[1°)]
%1-
	\item Calcule a soma dos 15 primeiros termos da PA $\left(2, \,5, \,8, \,11, ... \right)$.
%2-
	\item Calcule a soma dos termos da PA $\left(5, \,8, \,..., \,71 \right)$.
%3-	
	\item Determine a soma dos múltiplos de 3, compreendidos entre 10 e 100.
%4-	
	\item Determine a som dos termos da PA $\left(6,10,...,78 \right)$.
%5-	
	\item Determine a soma dos 30 primeiros números naturais.
%6-	
	\item Determine a soma dos múltiplos de 8, entre 6 e 100.
%7-	
	\item Um jardim tem uma torneira e 10 roseiras dispostas em linha reta. A torneira dista 50 metros da primeira roseira e cada roseira dista 2 metros da seguinte. Um jardineiro, para regar as roseiras, enche um balde na torneira e despeja seu conteúdo na primeira. Volta à torneira e repete a operação para cada roseira seguinte. Após regar a última roseira e voltar a torneira para deixar o balde, ele terá andado:
	\begin{enumerate}[a)]
		\item $1.200 m$
		\item $1.180 m$
		\item $1.110 m$
		\item $1.000 m$
		\item $1.130 m$
	\end{enumerate}
%8-
	\item Um atleta corre sempre 400 metros a mais do que no dia anterior. Ao final de 11 dias, ele percorreu um total de $35.200$ metros. O número de metros que ele percorreu no último dia foii igual a:
	\begin{enumerate}[a)]
		\item $5.100 m$
		\item $5.200 m$
		\item $5.300 m$
		\item $5.400 m$
		\item $5.500 m$
	\end{enumerate}
%9-	
	\item Encontre a PA de 3 termos cuja soma é 18 e o produto é 192.
%10-	
	\item Os termos da equação $5+x+...+30=105$ formam uma PA. Então o valor de $x$ é:
	\begin{enumerate}[a)]
		\item 6
		\item 15
		\item $\dfrac{15}{2}$
		\item 10
		\item $\dfrac{5}{2}$
	\end{enumerate}
%11-	
	\item Determine a soma dos 10 primeiros números naturais ímpares.
%12-	
	\item Calcule a medida dos lados de um triângulo retângulo, sabendo que eles formam uma razão 3.
%13-	
	\item Três números em progressão aritmética são tais que sua soma é 21 e seu produto, 231. Se $r$ é a razão dessa PA, determine $|r|$.
%14-	
	\item Na sequência $\left(\dfrac{1}{2},\, \dfrac{5}{8},\, \dfrac{3}{4},\, \dfrac{7}{8},\, x,\, y,\, z,\, ... \right)$ os valores de $x,\, y$ e $z$ são, respectivamente:
	\begin{enumerate}[a)]
		\item $\left(1, \dfrac{9}{8}, \dfrac{5}{4} \right)$
		\item $\left(\dfrac{1}{4}, \dfrac{3}{8}, \dfrac{5}{4} \right)$
		\item $\left(\dfrac{11}{4}, \dfrac{9}{8}, \dfrac{13}{4} \right)$
		\item $\left(\dfrac{5}{4}, \dfrac{9}{8}, \dfrac{7}{4} \right)$
		\item $\left(\dfrac{9}{4}, \dfrac{13}{8}, \dfrac{11}{4} \right)$
	\end{enumerate}
%15-	
	\item Um escritor escreveu, em um certo dia, as 20 primeiras linhas de um livro. A partir desse dia, elle escreveu, em cada dia, tantas linhas quantas havia escrito no dia anterior mais 5 linhas. O livro tem 17 páginas, cada uma com exatamente 25 linhas. Em quantos dias o escritor terminou de escrever o livro?
	\begin{enumerate}[a)]
		\item 8
		\item 9
		\item 10
		\item 11
		\item 17
	\end{enumerate}
%16-
	\item Interpolando 7 termos aritméticos entre os números 10 e 98 obtém-se uma progressão aritmética cujo termo central é:
	\begin{enumerate}[a)]
		\item 45
		\item 52
		\item 54
		\item 55
		\item 57
	\end{enumerate}
%17-	
	\item A soma dos $n$ primeiros termos de uma progressão aritmética é $n^2+2n$. O 10° termo dessa PA vale:
	\begin{enumerate}[a)]
		\item 17
		\item 18
		\item 19
		\item 20
		\item 21
	\end{enumerate}
%18-	
	\item A soma dos termos de uma PA é dada por $S_n=n^2-n,n=1,2,3,...$. Então, o 10° termo da PA vale:
	\begin{enumerate}[a)]
		\item 18
		\item 90
		\item 8
		\item 100
		\item 9
	\end{enumerate}
%19-	
	\item Se o termo geral de uma sequência é $a_n=5n-13$, com $n \in \mathbb{N}^{*}$, então a soma dos seus 50 primeiros termos é:
	\begin{enumerate}[a)]
		\item 5850
		\item 5725
		\item 5650
		\item 5225
		\item 5150
	\end{enumerate}
%20-	
	\item Calcule o valor não nulo de $x$, para que os números $x^2 +10, 9x, x-10$, nesta ordem, sejam termos consecutivos de uma progressão aritmética.
%21-	
	\item A soma dos $"n"$ primeiros termos de uma PA é dada por $4n^2$. Qual o valor do 8° termo dessa PA?
	
\end{enumerate}

\end{document}