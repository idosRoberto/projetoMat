\documentclass[a4paper,12pt, oneside]{book}

\usepackage[top=2cm, bottom=2cm, left=2.5cm, right=2.5cm]{geometry}

\usepackage[T1]{fontenc}
\usepackage[utf8]{inputenc}
\usepackage[brazilian]{babel}
\usepackage{setspace}
\usepackage{indentfirst}
\usepackage{graphicx}
\usepackage{subfigure}
\usepackage[nottoc]{tocbibind} %Pra colocar sa referências bibliográficas
% Os 3 pacotes abaixo são para o roda pé
\usepackage{ftnxtra}
\usepackage{fnpos}
\usepackage{lipsum}

\usepackage{makeidx} % Para índice remissivo

\linespread{1.5} %Espaçamento em todo o texto
\makeindex

\begin{document}

\tableofcontents
\listoffigures %Sumário para mostrar imagens
\listoftables %Sumário para mostrar tabela
\newpage

\chapter{Introdução}

%INSERINDO TABELA
\begin{table}[h]
\centering
\caption{Os Maiores Países Do Mundo Em Extensão \cite{IBGE}\footnote{Retirado Do Site Do IBGE}}
\vspace{0.5cm}
	\begin{tabular}{c|cc}
		Posição & Países & Extenção Territorial $Km^2$\\ \hline
		1 & Rússia & 17.089.246\\
		2 & Canadá & 9.984.670\\
		3 & China & 9.596.961\\
		4 & Estados Unidos & 9.371.174\\
		5 & Brasil & 8.515.767
	\end{tabular}
\end{table}
% O comando abaixo força a nota de roda pé a aparecer
\footnotetext{Retirado Do Site Do IBGE}

\section{Assunto 1}
\section{Assunto 2}

\chapter{Desenvolvimento}
\section{Assunto 3}

	\begin{figure}[h]
	\centering
	\includegraphics[scale=0.75]{C:/Users/rober/Pictures/Lampada.png}
	\caption{Imagem De Lâmpada\cite{FARIAS}\footnote{Imagem Retirad Da Internet}}\label{FigLampada}
	\end{figure}
\footnotetext{Imagem Retirad Da Internet}	

\vspace{2cm}

Possuo \index{Gradução} graduação \index{Graduação!Licenciatura} em Licenciatura em Ciências da Natureza e tecnologia em \index{Graduação!Análise e desenvolvimento de Sistemas} Análise e desenvolvimento de Sistemas.
	
\section{Assunto 4}

\chapter{Conclusão}
\section{Assunto 5}

	\begin{figure}[h]
		\centering
		\subfigure[Planeta Terra\label{FigTerra}]{\includegraphics[scale=1]{C:/Users/rober/Pictures/Terra.png}}
		\subfigure[Planeta Marte\label{FigMarte}]{\includegraphics[scale=1]{C:/Users/rober/Pictures/Marte.png}}
		\subfigure[Planeta Saturno\label{FigSaturno}]{\includegraphics[scale=1]{C:/Users/rober/Pictures/Saturno.png}}
		\caption{Alguns Planetas Do Sistema Solar}\label{FigSistema}
		
	\end{figure}

\section{Assunto 6}

Aprenda, persista, e estude a cada dia \cite{FARIAS}


\begin{thebibliography}{2}

	\bibitem{FARIAS} FARIAS, J.R; Curso De Latex. Universidade Federal De Sergipe, 2023 
	\bibitem{IBGE} IBGE, Instituito Brasileiro De Pesquisa E Estatística, 2023

\end{thebibliography}

\printindex
\end{document}